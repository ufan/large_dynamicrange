\documentclass[]{article}

%opening
\title{Reply to reviewer's comments}
\author{Yong Zhou}

\begin{document}

\maketitle

\section{}
Comment 1: First of all, let me comment on the title of the submitted paper. More than a "read-out unit", the paper simply describe the "read-out" of the PSD detector. There is no "unit" mentioned in the paper. I suggest to change the title accordingly.

Comment 2: The paper needs careful english language revisions. Lot of errors, bad phrasing, inaccurate wording can be found throughout the text (se, e.g., line 1 "Paricle", also in the abstract, line 3 "study" instead of "studies", line 14 "lights" instead of "light", line 16 "these" instead of "this", etc.). The authors should have the manuscript reviewed by an english mother-tongue scientist.

Comment 3: Line 40: The reference to AMS-02 test beam results is a bit outdated. More recent data can be found in V. Bindi et al., "Calibration and performance of the AMS-02 time of flight detector in space", NIM-A, 743 (2014) 22-29. I suggest the authors to use formula (12) and Fig. 11 therein to describe the light yield saturation for heavy nuclei.

Comment 4: Figures 4, 5, 6 and 7: Authors should avoid to put text in the figure. Details about curves and/or markers should be put in the caption.

Comment 5: Table 1: Parameters are poorly described. Measurements with 40Ar refer to DY5, measurements with MIPs refers to DY8; Calibration is (ADC channels)/fC, etc.s

%%%%%%%%%%%%%%%%%
First, we have made some modifications based on the review's comments, listed point-by-point as follows:
1. The title has been changed to "A large dynamic range readout design for the plastic scintillator detector of DAMPE".

2. The article has underwent a careful language revision. Errors have been corrected, inaccurate words and bad phrases have been replaced, and ambiguous expressions and sentences have been modified or reorganized to clarify the opinion more clearly. 
Most of these modifications are trivial and can clarify themselves in the context. Here only the major modifications are summarized:
     (1) To describe the four factors affecting the dynamic range more clearly (i.e. one paragraph for light yield response, one paragraph for the traversing length, one paragrah for the light yield fluctuation and one paragraph for the light attenuation), the paragraphs in Section 2 are re-organized: the original second and third paragraph of Section 2 are combined into one paragraph, because they're both talking about the light yield response; and the original 4th paragraph of Section 2 are split into two paragraphs, one for the transversing length and the other one for the light yield fluctuation.  
     (2) In the first paragraph of Section 3.1: added a sentence about the VA160 composition to help readers understand the charge measurement process.
     (3) An error in Eq (3) has been corrected, i.e. from "Qn-1 - Qn" to "Qn - Qn-1".
     (4) An error in Eq (5) has been corrected: added a minus sign, i.e. from "(2L-x)" to "-(2L-x)"
     (5) The original second paragraph of Section 4.2 was split into two paragraphs to make the description more concise: one for presenting the beam test results and the other one for the analysis based on beam test results.  
     (6) The original Table 1 is split into 3 new tables (also see the reply to Comment 5). To describe the parameters in these tables more precisely and to present the test results more clearly, the following revisions are made:
	     i) in the first paragraph of Section 4: added the description about the electronic calibration to help explain the parameters in the new Table 1.
	     ii) in the second paragraph of Section 4.1: added a sentence about the test configuration in the center; added a sentence explaining the reason for using Dy8 to display the MIP response; also added the formulas for the lower limit and Dy8 covering range calculation and the calculation results are presented for both sides instead of one.
	     iii) in the 4th paragraph of Section 4.1: added the formula for the upper limit calculation and the results are presented for the two sides of the PSD module instead of one. The results are slightly larger than the previous ones, because we have incorporated the calibration parameter into the calculation to covert the ADC counts into charge. The calibration parameter was ignored by mistake in the previous calculation. So the new results are more accurate than the previous ones.
	     iv) in the new third paragraph of Section 4.2: added the formula for calculating the light yield of Ar and the results are presented for both sides instead of one; the largest signal of Ca in PSD bar is also re-calculated and the results are presented for both sides instead of one.
      

3. The recommended article (NIM-A, 743 (2014) 22-29) is an excellent resource and the formula (12) in this article do give us a new sight on the modeling of the heavy ion response. However, the article does not list the values of the fitting parameters of formula (12). And the Y-axis of Fig. 11 uses log coordinate which brings large errors when determining the data points by hand. We can't get the exact values needed for the dynamic range estimation based on this article. So, our estimation in the paper is still based on the old article, but we have added the new article as a reference.

4. Description texts about the curves and marks have been removed from the these figures and put into the caption accordingly. But the statistical/fitting boxes of the histograms are retained because they are usually considered as part of the figure.

5. To make it more clear, the original Table 1 has been split into three new tables based on their contents: a table for the electronic calibration result, a table for the cosmic ray test result and a table for the Ar beam test result. Each parameter has also been represented by the same dedicated symbol in the tables and in the text to help the description. Detailed description about these parameters have been added in the corresponding paragraphs: new Table 1 is referenced in the first paragraph in Section 4, new Table 2 is referenced in Section 4.1 and new Table 3 is referenced in Section 4.2. 

\end{document}
