\documentclass[]{article}

%opening
\title{Reply to reviewer's comments}
\author{Yong Zhou}

\begin{document}

\maketitle

\section{}
Comment 1: First of all, let me comment on the title of the submitted paper. More than a "read-out unit", the paper simply describe the "read-out" of the PSD detector. There is no "unit" mentioned in the paper. I suggest to change the title accordingly.

Comment 2: The paper needs careful english language revisions. Lot of errors, bad phrasing, inaccurate wording can be found throughout the text (se, e.g., line 1 "Paricle", also in the abstract, line 3 "study" instead of "studies", line 14 "lights" instead of "light", line 16 "these" instead of "this", etc.). The authors should have the manuscript reviewed by an english mother-tongue scientist.

Comment 3: Line 40: The reference to AMS-02 test beam results is a bit outdated. More recent data can be found in V. Bindi et al., "Calibration and performance of the AMS-02 time of flight detector in space", NIM-A, 743 (2014) 22-29. I suggest the authors to use formula (12) and Fig. 11 therein to describe the light yield saturation for heavy nuclei.

Comment 4: Figures 4, 5, 6 and 7: Authors should avoid to put text in the figure. Details about curves and/or markers should be put in the caption.

Comment 5: Table 1: Parameters are poorly described. Measurements with 40Ar refer to DY5, measurements with MIPs refers to DY8; Calibration is (ADC channels)/fC, etc.s

%%%%%%%%%%%%%%%%%
First, we have made some modifications based on the review's comments, listed point-by-point as follows:
1. The title has been changed to "The large dynamic range readout design of the plastic scintillator detector of DAMPE".

2. 

3. The recommended article (NIM-A, 743 (2014) 22-29) is an excellent resource and the formula (12) in this article do give us a new sight on the modeling of the heavy ion response. However, the article does not list the values of the fitting parameters of formula (12). And the Y-axis of Fig. 11 uses log coordinate which brings large errors when determining the data points by hand. So we can't get the exact values needed for the dynamic range estimation based on this article. Our estimation in the paper is still based on the old article, but we have add the new article as a reference.

4. Description texts about the curves and marks have been removed from the these figures and put in the caption accordingly. But the statistical/fitting boxes of the histograms are retained because they are usually considered as part of the figure.

5. To make it more clear, the original Table 1 has been split into three new tables based on their contents: a table for the electronic  calibration result, a table for the cosmic ray test result and a table for the Ar beam test result. Description about these parameters have been added in the corresponding sections. Each parameter has also been represented by a dedicated symbol to help description.

\end{document}
